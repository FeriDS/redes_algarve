\documentclass{article}
\usepackage{hyperref}

\usepackage{xcolor,enumerate,hyperref}
\input{handout}

\definecolor{darkgreen}{rgb}{0,0.8,0.3}
\definecolor{darkblue}{rgb}{0.00,0.00,0.55}

\begin{document}

\homework{Redes II}{DueDateGoesHere}{Lab \#4{}}

\section*{Myths and truths about IPv6}

IPv6 is still widely misunderstood among engineers across the globe. From a security
standpoint, IPv4 and IPv6 are very similar. Let's try to dispel the most common security
myths regarding IPv6. Decide whether each of the following are myths or truths. Justify
your answer in each case.

  \begin{enumerate}
  \item \textbf{Myth 1}: ``A Man-in-the-Middle attack is impossible with Ipv6''.
    Check this
    \href{https://www.eweek.com/security/attackers-can-use-ipv6-to-launch-man-in-the-middle-attacks}{link}. \\ \smallskip

    \textcolor{darkblue}{%
      Write your answer here.
      }
    
    \medskip
    
  \item \textbf{Myth  2}: ``IPv6 Security enhancements (such as IPsec) make it
    safer than IPv4''. Check this
    \href{http://www.ipv6now.com.au/primers/IPv6PacketSecurity.php}{link}. \\ \smallskip 

     \textcolor{darkblue}{%
      Write your answer here.
      }
    
    
    
    \medskip
    %
  \item \textbf{Myth 3}: ``IPv4 is more secure than IPv6 because of NAT''. Check this \href{https://security.stackexchange.com/questions/44065/with-ipv6-do-we-need-to-use-nat-any-more}{link}. \\ \smallskip 

     \textcolor{darkblue}{%
      Write your answer here.
      }
    
    
    
    \medskip
    
  \item \textbf{Myth  4}: ``Due to large address space in IPv6, address
    scanning attacks are impossible''. Check this \href{https://www.internetsociety.org/blog/2015/02/ipv6-security-myth-4-ipv6-networks-are-too-big-to-scan/}{link}. \\ \smallskip 

     \textcolor{darkblue}{%
      Write your answer here.
      }
    
    
    
    \medskip

  \end{enumerate}

 

\end{document}
